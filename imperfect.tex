In this chapter we generalize our problem to allow for partitioning $C$ into parts
that are not necessarily clone sets.
We then measure how much such a partition differs from a decloning.
For given preference profile we try to find a partition that minimizes this value.

\section{General Problem}

The problem of \textsc{Decloning} turned out to be simple in the sense of $P$ vs. $NP$-hard.
We may suppose that it is caused by too tight requirements imposed on a part of candidate set partition,
namely it being a clone set.
Now we drop this restriction and instead measure how different in some sense a part is from some clone set.
Of course, we require that this measure equals $0$ only for clone sets.

\begin{defn}[clone set error function]
Given a preference profile $P$ over $C$ we say that a function
$f: 2^C \setminus \{\emptyset\} \rightarrow \mathbb{R}_+\cup\{0\}$
is a clone set error function for $P$ when $f(X) = 0$ iff $X$ is a clone set for $P$.
\end{defn}

Now we can generalize the \textsc{Decloning} problem into \textsc{OptimalDecloning}
where we try to minimize the sum of errors of parts of candidate set partition.


\begin{problem}{OptimalDecloning}
% TODO: function f as a polynomial circuit
	Input: Preference profile $P$ over $C$, clone set error function $f$,
		integer $k$, positive real number $l$.

	Question: Is there a partition of $C$ into $C_1, ..., C_k$ such that $\sum_{i=1}^k f(C_i) \leq l$?
\end{problem}

\begin{thm} \label{optdecl}
	\textsc{OptimalDecloning} is $NP$-complete.
\end{thm}

\begin{lmm} \label{lmmnotrivial}
	Preference profle $P = (\succ_1, ... , \succ_m)$ over $C = \{1, ..., n\}$
	where for each $i=1,...,m$ we have
	$$i \succ_i i+1 \succ_i ... \succ_i n \succ_i 1 \succ_i 2 \succ_i ... \succ_i i-1$$
	has no non-trivial clone sets.
\end{lmm}

\begin{proof}[Proof by contradiction]
Let us consider a non-trivial clone set $X$ for $P$.
It is a contiguous subsequence of preference order $\succ_1$, so $X = \{i,...,j\}$ for some $i,j \in C$.
Clone set $X$ is not trivial, so both $i\neq{j}$ and $X\neq{C}$.
Let us notice that $X$ is not contiguous on the preference order $\succ_{i+1}$.
Subset $X$ is thus not a clone set for $P$.
\end{proof}

\begin{proof}[Proof of theorem \ref{optdecl}]
We will show a reduction from the \textsc{X3C} problem.

We are given set $X$, collection $S$ and integer $n$.
Let us take clone set $C = X$ and function $f:2^C\rightarrow \{0,1,n+1\}$ such that:
$$ f(X) =
\begin{cases}
0, &\text{when} \abs{X} \in \{1,3n\} \\
1, &\text{when} X \in S \\
n+1, &\text{in other cases}
\end{cases}
$$
Function $f$ is well-defined when $n>1$, because if $X \in S$ then $\abs{X} = 3 \not\in \{1,3n\}$.
Condition $n=1$ can be special-cased during the runtime of reduction.
Let us take preference profile $P$ over $C$ as in lemma \ref{lmmnotrivial}.
Function $f$ is a clone set error function for $P$, because $f(X) = 0$ iff $X$ is a trivial clone set
and preference profile $P$ has no non-trivial clone sets.
Let us take $l=k=n$, what gives us all the inputs necessary for \textsc{OptimalDecloning} problem.
Let us check when the answer to this problem is \textsc{yes}.

No part $X$ of the partition of $C$ can have $f(X)=n+1$, as it itself exceeds limit $l$.
This means that every part $X$ either belongs to $S$ or has $\abs{X} \in \{1,3n\}$.
In both cases $\abs{X} \in \{1,3,3n\}$.
When $n>1$ we cannot use $3n$-sized parts,
as it yields only one part in partition while $k>1$.

We are left with $\abs{X} \in \{1,3\}$.
Let us note that due to the specified size $k=n$ of partition and $\abs{C} = 3n$
the partition can contain only three-element parts.
This means that we may choose $X$ only from $S$.
It is possible to construct a valid partition only when there is an $A \subset S$ satisfying the \textsc{X3C} problem.

In that case we have partition $A$ of candidate set $C$ such that
$$ \sum_{X \in A} f(X) = \sum_{X \in A} 1 = n,$$
so the answer to \textsc{OptimalDecloning} problem is \textsc{yes}
iff the answer to \textsc{X3C} is~\textsc{yes}.
\end{proof}



\section{Specific Error Functions}

Let us try to fix a value of error function input to \textsc{OptimalDecloning} problem
treating it as if it was hardcoded into the specification of the new problem.

Given a specific error function we will consider the following family of problems.

\begin{problem}{Specific Error Function $f$ Optimal Decloning Template}
	Input: Preference profile $P$ over $C$, integers $k$ and $l$.

	Question: Given than $f$ is a known component error function for $P$,
		is there a partition of $C$ into $C_1, ..., C_k$ such that $\sum_{i=1}^k f(C_i) \leq l$?
\end{problem}

We want from such an error function $f$ to capture the intuitive notion
of distance of given partition from some decloning.

\subsection{Component Error Function}

% intuition for f_com

To formulate the definition of component error function we will need a definition of kernel of total order relation.

Given a total order $\succ$ over a finite set we may sort its elements into a sequence
such that for every pair of elements former is in the relation $\succ$ with the latter.
This sequence naturally induces another relation, namely relation $\overline{\succ}$
between every two consecutive elements.
We will call such a relation a kernel of $\overline{\succ}$.
It is easy to recover $\succ$ from $\overline{\succ}$ simply by taking the transitive closure of $\succ$.

\begin{defn}[kernel of total order]
Given a total order $\succ$ over a finite set we say that $\overline{\succ}$ is its kernel when
$\overline{\succ}$ is a minimal relation whose transitive closure equals $\succ$.
\end{defn}

\begin{exmp}
Kernel of natural order $>$ over set $\{1,2,3,4\}$ is the relation
$\overline{>} = \{(4,3), (3,2), (2,1)\}$.
\end{exmp}

\begin{defn}[component error function]
Let us define for given preference profile $P = (\succ_1, ..., \succ_m)$ over candidate set $C$
a function $f_{com}: 2^C \setminus \{\emptyset\} \rightarrow \mathbb{N}_0$ with the following formula:
$$ f_{com}(X) = \sum_{i=1}^m \left( \abs{X}-1 -
\abs{\left\lbrace \left( a,b \right) \in X^2 : a \overline{\succ}_i b \right\rbrace }
\right) \text{.}$$
\end{defn}

\begin{rmrk}
Function $f_{com}$ for preference profile $P$ is a clone set error function for $P$.
\end{rmrk}

\begin{problem}{ComponentOptimalDecloning (COD)}
	Input: Preference profile $P$ over $C$, integers $k$ and $l$.

	Question: Given than $f_{com}$ is a component error function for $P$,
		is there a partition of $C$ into $C_1, ..., C_k$ such that $\sum_{i=1}^k f_{com}(C_i) \leq l$?
\end{problem}

Given a preference profile $P = (\succ_1, ..., \succ_m)$ let us create now a multigraph $G=(C,E)$
where $E$ is a multiset obtained by multisum
of each $\{a,b\}$ such that $a,b \in P$ and $a \overline{\succ} b$ for some $\succ \in P$.
We get a multigraph created from Hamiltonian path of each voter's preferences.
Let us call this a Hamiltonian path multigraph.

\begin{defn}[Hamiltonian path multigraph]
A multigraph $G=(V,E)$ is a Hamiltonian path multigraph if there exists a partition of $E$
into sets of edges of Hamiltonian paths in $G$.
\end{defn}

Partitioning $C$ into $C_1,...,C_k$ corresponds to
partitioning Hamiltonian path multigraph $G$ into submultigraphs $G_1,...,G_k$
% necessary induced graphs? maybe just partition of V?
induced by $C_1,...,C_k$ respectively.
Let us denote by $c_{ij}$ part
$\abs{\left\lbrace \left( a,b \right) \in C_i^2 : a \overline{\succ}_j b \right\rbrace }$
of the formula of $f_{com}(C_i)$ such that
$f_{com}(C_i) = \sum_{j=1}^m \left( \abs{C_i} - 1 - c_{ij} \right)$.
We have that $c_{ij}$ is the number of edges from $j$-th voter's Hamiltonian path
that connects two vertices from $C_i$.
We will call such edges the $C_i$ intra-part edges and edges that end in different parts the inter-part edges.

Now the error value equals:
$$
\sum_{i=1}^k f_{com}(C_i) =
\sum_{i=1}^k \sum_{j=1}^m \left( \abs{C_i} - 1 - c_{ij} \right) =
m \sum_{i=1}^k \abs{C_i} - m k - \sum_{i=1}^k \sum_{j=1}^m c_{ij} =
m(n-k) - \sum_{j=i}^m \sum_{i=1}^k c_{ij}
\text{.}$$

Let us note that $\sum_{i=1}^k c_{ij}$ is the total number of $C_i$ intra-part edges
and $\sum_{j=i}^m \sum_{i=1}^k c_{ij}$ is the total number of intra-part edges.
As value of $m(n-k)$ is constant for a given graph,
we have that \textsc{ComponentOptimalDecloning} is essentially a problem of maximizing the number of intra-part edges.
We will that number a score of partition.

\begin{defn}[partition score]
Given a multigraph $G=(V,E)$, $E=(V,e)$ and a partition $(V_1, ..., V_k)$ of $V$
we say that a score of part $V_i$ for $i=1,...,k$ equals $\sum_{u,v \in V_i} e(\{u,v\})$.
We say that a partition score is a sum of the scores of its parts.
\end{defn}

\begin{rmrk}
Score of a part that is a singleton equals $0$.
\end{rmrk}

This lets us to formulate another problem which is equivalent to \textsc{ComponentOptimalDecloning}
in terms of reduction.

\begin{problem}{HamiltonianPathMultigraphPartition (HPMP)}
	Input: Hamiltonian path multigraph $G=(V,E)$
		together with its partition into Hamiltonian paths, integers $k$ and $s$.

	Question: Is there a $k$-sized partition of $V$ of score at least $s$?
\end{problem}

\begin{rmrk}
There is a reduction of problem \textsc{HPMP} to \textsc{COD}.
\end{rmrk}

\begin{thm} \label{ultimate-theorem}
\textsc{ComponentOptimalDecloning} is $NP$-complete.
\end{thm}

\begin{lmm} \label{hpmd-red}
For every graph $G=(V,E)$ and hamiltonian path multigraph $G=(V',E')$, $E=(V',w)$ such that $V' \supset V$ and
$$ w(\{u,v\}) =
\begin{cases}
2, 				&\text{when } \{u,v\} \in E \\
0, 				&\text{when } \{u,v\} \not\in E \text{ and } u,v \in V \\
w_{u v}\leq 1,	&\text{when } u \not\in V \text{ and } v \not\in V
\end{cases}
$$
there is a clique of size $c$ in $G$ iff
there is $\left(\abs{V'}-c-1\right)$-sized partition of $V'$ of score $c(c-1)$.
\end{lmm}

\begin{proof} \label{hpmp-red-2}
Let us consider any graphs $G=(V,E)$ and $G'=(V',(V',w))$ satisfying the condition of the lemma.
If there is a $c$-sized clique $C \subset V$ in $G$,
then we may partition $V'$ into $C$ and singletons for each $u \in V' \setminus C$.
As $f(\{u,v\})=2$ for $u,v \in C$, such a partition has score $2 {c(c-1) \over 2} = c(c-1)$.
This proves the implication in one direction.

Now let us study the case when there is no $c$-sized clique in $G$.
Let us consider any partition $P$ of $V'$, $P=(V_1, ..., V_k)$ for $k = \abs{V'}-c-1$.
Partition $P$ cannot contain any part bigger than $c$,
because all of the parts need to be non-empty.
If $P$ contains $c$-sized part $V_1$ included entirely in $V$, then its score is less than $c(c-1)$,
for $V$ does not contain $c$-sized cliques.
In other case part $V_1$ contains a vertex from $V' \setminus V$.
This means that there is an edge $e$ ended in $V_1$ such $w(e)<=1$, so the score of $P$ is less that $c(c-1)$.

The only case left is when each part of $P$ has size less than $c$.
We will do reasoning similar to the one in the proof of theorem \ref{gp-np}.
Let us denote the upper bound of the score $m$-sized part of $f(m)$, $f(m) = m(m-1)$.
Function $f$ is strictly convex because $f'' = 2 > 0$
and sequence $(c,1,...,1)$ majorizes sequence $S$ of sizes, so we can use inequality \ref{thm:Kar}
for $f$ and both sequence.
This gives us the following upper limit on score $q$ of $P$:
$$q \leq \sum_{s \in S} f(s) < f(c) + f(1) + ... + f(1) = c(c-1) \text{.}$$
So also in this case the score of $P$ is less than $c(c-1)$ and the lemma is proved.
\end{proof}

\begin{lmm}
Given a graph $G$ and integer $c$ there is a polynomial-time construction
of hamiltonian path multigraph $G'$ together with its path decomposition
satisfying the conditions of lemma \ref{hpmd-red}.
\end{lmm}

\begin{proof}
\end{proof}

\begin{proof}[Proof of theorem \ref{ultimate-theorem}.]
We will show a reduction from \textsc{Clique} problem to \textsc{HPMP}.
We are given graph $G$ and integer $c$
Lemma \ref{hpmp-red-2} shows a polynomial-time construction of multigraph $G'$.
Let us take $k=\abs{V'}-c-1$ and $s=c(c-1)$.
Lemma \ref{hpmd-red} proves that it is a valid reduction, i.e. it preserves the answer.
So \textsc{HPMP} problem is $NP$-complete.
\end{proof}


\subsection{Inconsistent Voters Error Function}

Component error function has useful properties, but also could yield partition
intuitively far from optimal.
It constructs the partition using only very local information,
such as edges in the graph of kernels of preference orders.
For example, in the proof of theorem \ref{ultimate-theorem} we constructed a partition
which was not in accordance with any voter.

On the other side of spectrum, we may think of an error function which
takes into consideration only global information in the sense of a single voter.
Such an information is what we call consistency with a voter's preferences, what we define below.

\begin{defn}[inconsistent voter]
We say that a voter described by preference order~$\succ$ over $C$
is consistent with a non-empty subset $X \subset C$
iff $X$ is a clone set for preference profile $P = (\succ)$.
We also say that $\succ$ is inconsistent with $X$ if it is not consistent with $X$.
\end{defn}

\begin{defn}[inconsistent voters error function]
Given a preference profile $P$ over $C$ we define function
$f_{iv}: 2^C \setminus \{\emptyset\} \rightarrow \mathbb{N}_0$ so that
$f(X)$ is a number of preference orders in $P$ that are inconsistent with $X$.
\end{defn}

\begin{rmrk}
Function $f_{iv}$ for preference profile $P$ is a clone set error function for $P$.
\end{rmrk}

Now we can introduce optimal decloning problem using $f_{iv}$ clone set error function.

\begin{problem}{ConsistentVotersOptimalDecloning (CVOD)}
	Input: Preference profile $P$ over $C$, integers $k$ and $l$.

	Question: Given than $f_{iv}$ is a inconsistent voters error function for $P$,
		is there a partition of $C$ into $C_1, ..., C_k$ such that $\sum_{i=1}^k f_{iv}(C_i) \leq l$?
\end{problem}

It remains an open problem whether \textsc{CVOD} problem is \np-hard.