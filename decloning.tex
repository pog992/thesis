In this chapter we formalize notions connected to elections and clones.
We also introduce the problem of detecting clones and show an efficient algorithm for that.

\section{Elections} \input{"elections.tex"}

\section{Decloning}

We are interested in a situation when some candidates are indistinguishable for voters.
In that case every voter ranks them consecutively.

\begin{defn}[clone set]
We say that non-empty $X \subseteq C$ is a clone set for preference profile $P = (\succ_1, ... , \succ_n)$ over $C$,
if for each $k,l \in X$, each $a \in C \backslash X$ and each $1 \leq i \leq n $ it holds that
$k \succ_i a$ iff $l \succ_i a$.
\end{defn}

\begin{exmp} \label{clone-sets}
Let us consider a candidate set $C = \{a,b,c,d,e,f\}$
and a preference profile $P = (\succ_1, \succ_2, \succ_3, \succ_4)$ defined by the following relations:
\begin{align*}
a \succ_1 b \succ_1 c \succ_1 d \succ_1 e \succ_1 f, \\
b \succ_2 c \succ_2 a \succ_2 d \succ_2 f \succ_2 e, \\
e \succ_3 f \succ_3 c \succ_3 b \succ_3 a \succ_3 d, \\
e \succ_4 f \succ_4 e \succ_4 a \succ_4 c \succ_4 b.
\end{align*}
In this case we have the following clone sets:
$$\{a,b,c,d,e,f\}, \{a\}, \{b\}, \{c\}, \{d\}, \{e\}, \{f\}, \{a,b,c\}, \{e,f\}, \{b,c\}.$$
\end{exmp}

\begin{rmrk}
Let us notice that for any preference profile $P$ over candidate set $C$
we have that each $c \in C$ and $C$ itself is a clone set.
\end{rmrk}

\begin{defn}[trivial clone set]
Given a candidate set $C$ we say that $X$ is a trivial clone set
when $X \in C$ or $X = C$.
\end{defn}

\begin{exmp}
In the example \ref{clone-sets} the only non-trivial clone sets were
$\{a,b,c\}$, $\{e,f\}$ and $\{b,c\}$.
\end{exmp}

\begin{defn}[decloning]
We call a decloning for preference profile $P$ over $C$
such a partition of $C$  into $C_1, ... , C_k$
that for each $1 \leq i \leq k$ $C_i$ is a clone set for $P$.
\end{defn}


\begin{problem}{Decloning}
    Input: preference profile $P$, integer $k$

    Question: Is there a decloning for $P$ with cardinality $k$?
\end{problem}

%$$
%pos(i) =
%    \begin{cases}
%        \top & \text{if $i=0$ or } \exists X \in clones(i): p(i - \card{X}) \\
%        \bot & \text{otherwise.}
%    \end{cases}
%$$

\begin{thm}
\textsc{Decloning} is in $P$.
\end{thm}