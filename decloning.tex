In this chapter we formalize notions connected to elections and clones.
We also introduce the problem of detecting clones and show an efficient algorithm for it.

\section{Elections} \input{"elections.tex"}

\section{Decloning}

We are interested in a situation where some candidates are indistinguishable for voters.
We expect a total order over the set of candidates for each voter,
so every voter needs to somehow break ties.
Still, members of a single clone set are ranked consecutively by every voter.

\begin{defn}[clone set]
We say that a non-empty set $X \subseteq C$ is a clone set
for preference profile $P = (\succ_1, ... , \succ_n)$ over $C$,
if for each $k,l \in X$, each $a \in C \backslash X$, and each $i=1, ..., n$ it holds that
$k \succ_i a$ if $l \succ_i a$.
\end{defn}

\begin{exmp} \label{clone-sets}
Let us consider a candidate set $C = \{a,b,c,d,e,f\}$
and a preference profile $P = (\succ_1, \succ_2, \succ_3, \succ_4)$ defined by the following relations:
\begin{align*}
a \succ_1 b \succ_1 c \succ_1 d \succ_1 e \succ_1 f, \\
b \succ_2 c \succ_2 a \succ_2 d \succ_2 f \succ_2 e, \\
e \succ_3 f \succ_3 c \succ_3 b \succ_3 a \succ_3 d, \\
d \succ_4 f \succ_4 e \succ_4 a \succ_4 c \succ_4 b.
\end{align*}
In this case we have the following clone sets:
$$\{a,b,c,d,e,f\}, \{a\}, \{b\}, \{c\}, \{d\}, \{e\}, \{f\}, \{a,b,c\}, \{e,f\}, \{b,c\}.$$
\end{exmp}

\begin{rmrk}
We note that for any preference profile $P$ over candidate set $C$
we have that for each $c \in C$, $\{c\}$ is a clone set and as is $C$ itself.
\end{rmrk}

\begin{defn}[trivial clone set]
Given a candidate set $C$, we say that $X$ is a trivial clone set
when $\abs{X} = 1$ or $X = C$.
\end{defn}

\begin{exmp}
In Example \ref{clone-sets} the only non-trivial clone sets are
$\{a,b,c\}$, $\{e,f\}$ and $\{b,c\}$.
\end{exmp}

\begin{defn}[decloning]
Let $P$ be a preference profile over candidate set $C$.
We say that a partition of $C$ into $C_1, ..., C_k$ is a decloning
if for each $i=1,...,k$, $C_i$ is a clone set for $P$.
\end{defn}

\begin{exmp}
In Example \ref{clone-sets}, some of the possible declonings are:
$$
\big\{ \{a,b,c\}, \{d\}, \{e,f\} \big\} \text{, }
\big\{ \{a\}, \{b,c\}, \{d\}, \{e\}, \{f\} \big\} \text{, }
\big\{ \{a,b,c,d,e,f\} \big\} \text{.}
$$
\end{exmp}

\begin{rmrk} \label{sequence-split}
We note that a decloning corresponds to some splitting into parts
the sequence from the preference order of the $i$-th voter for $i=1,...,n$ (e.g. the first voter).
\end{rmrk}


\begin{problem}{Decloning}
    Input: preference profile $P$ for candidate set $C$, integer $k$

    Question: Is there a decloning for $P$ with cardinality $k$?
\end{problem}

\begin{thm}
\textsc{Decloning} is in $P$.
\end{thm}

\begin{proof}
We will show a polynomial algorithm for the \textsc{Decloning} problem.

Let $C = \{c_1, ..., c_n\}$ be out input candidate set
and let $P = (\succ_1, ... , \succ_m)$ be the input preference profile.
Without loss of generality, let us assume that $c_1 \succ_1 c_2 \succ_1 ... \succ_1 c_n$.
We will use Remark \ref{sequence-split} to split the sequence $(c_1, ..., c_n)$
into $k$ parts, to obtain a decloning of $P$.

Let us define a predicate
$\mathit{IsCloneSet}: 2^C \rightarrow \{\text{\textsc{true}},\text{\textsc{false}}\}$
such that $\mathit{IsCloneSet}(X) = \text{\textsc{true}}$ iff $X$ is a clone set for $P$.
It is easy to see that $\mathit{IsCloneSet}$ is computable in polynomial time
by an explicit application of the definition of a clone set.

Now let us define another predicate,
$\mathit{IsDeclonable}: \{1,...,n\}\times\{1,...,k\} \rightarrow \{\text{\textsc{true}},\text{\textsc{false}}\}$
such that $\mathit{IsDeclonable}(i,j) = \text{\textsc{true}}$
iff there is a decloning for $P$ over reduced candidate set $\{c_1,...,c_i\}$
with cardinality $j$.
We note that $\mathit{IsDeclonable}(i, 1)$ for every valid $i$, because partition to one part is a decloning.
We also note that for $j>1$ it holds that $\mathit{IsDeclonable}(i,j)$ iff
there exists $1 \leq a < i$ such that $\mathit{IsDeclonable}(a,j-1)$ and $\mathit{IsCloneSet}(\{c_{a+1}, ..., c_i\})$.
Thus predicate $\mathit{IsDeclonable}$ can be computed with the following recursive formula:
$$ \mathit{IsDeclonable}(i,j) =
\begin{cases}
\text{\textsc{true}},	&\text{when $j=1$} \\
\mathit{IsCloneSet}(\{c_i\}) \wedge \mathit{IsDeclonable}(i-1,j-1) \vee \\
\mathit{IsCloneSet}(\{c_{i-1},c_i\}) \wedge \mathit{IsDeclonable}(i-2, j-1) \vee \\
... \vee \\
\mathit{IsCloneSet}(\{c_2,...,c_i\}) \wedge \mathit{IsDeclonable}(1,j-1)
, 		&\text{when $j>1$}
\end{cases}
$$

Direct implementation of this recursion with memoizing
results in a polynomial-time algorithm that computes the value of $IsDeclonable(n,k)$
solving the \textsc{Decloning} problem.
\end{proof}
