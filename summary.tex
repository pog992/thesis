\section{Results}

In this work we studied decloning, which is a partition of candidates into clone sets.
We studied the \textsc{Decloning} problem of finding,
for a given preference profile, a decloning of given cardinality.
We have shown that the \textsc{Decloning} problem is in \p.
This means that if all voters perfectly detect clones
and there is no noise in their preference orders,
we can easily infer clones from voters' preferences.

We have also considered imperfect clones.
We have measured clone-wise imperfection of a given subset of candidates using clone set error functions.
We have defined the \textsc{OptimalDecloning} problem.
It takes as input any polynomial-time computable clone set error function and preference profile,
and computes partition of the candidate set with the lowest possible value of the sum of clone set errors for each part.
We have also defined the \textsc{ComponentOptimalDecloning} problem which uses a fixed clone set error function.
This error function computes how much extra parts a candidate set subset is split into
on preference orders of voters.

We have shown that problems \textsc{OptimalDecloning} and \textsc{ComponentOptimalDecloning}
are \np-complete.
This means that it is potentially intractable to detect clones in elections,
in conditions where votes contain noise or are randomly biased.

We have also considered problem \textsc{ConsistentVotersOptimalDecloning}
which uses a different clone set error function than \textsc{ComponentOptimalDecloning}.
Its error function returns how many voters has to be removed from the elections
to make the given candidate subset a clone set.

Table \ref{cclasses} summarizes complexity classes of studied decloning problems.

\begin{table}
\centering
\begin{tabular}{| c | c |}\hline
	\textbf{problem} & \textbf{complexity class} \\ \hline
	\textsc{Decloning} & \p \\ \hline
	\textsc{OptimalDecloning} & \np-complete \\ \hline
	\textsc{ComponentOptimalDecloning} & \np-complete \\ \hline
	\textsc{ConsistentVotersOptimalDecloning} & \np \\ \hline
\end{tabular}
\caption{Complexity classes of considered problems.} \label{cclasses}
\end{table}


\section{Future Research}

Natural continuation of this work is to more accurately classify the complexity of \textsc{ConsistentVotersOptimalDecloning} problem.
We may expect that, similarly to \textsc{ComponentOptimalDecloning}, this problem is \np-complete.

Moreover, \np-completeness of optimal decloning problems does not mean
that decloning in face of voters' errors is computationally infeasible.
There may be either efficient approximation algorithms
or parametrizations resulting in fixed-parameter tractable algorithms.

It is also possible that there is a natural clone set error function
for which optimal decloning problem is in \p.

Another interesting result would be to show a dichotomy that determines
which clone set error functions yield optimal decloning problem in \p
and which yield problems that are \np-hard.
