\section{Results}

In this work we defined elections, decloning and decloning problems.

We have shown that \textsc{Decloning} problem is in \p.
This means that if all voters perfectly detect clones
and there is no noise in their decision,
we can easily infer clones from voters' preferences.

We have also shown that problems \textsc{OptimalDecloning} and \textsc{ComponentOptimalDecloning}
are \np-complete.
Those problems are reconstructing optimal partition minimizing 
sum of distances to some clone sets for each part.
They use any metric to measure distance to some clone set and a fixed one respectively.
This means that it is potentially intractable to detect clones in elections
in conditions where votes contain noise or are randomly biased.

We have also considered problem \textsc{ConsistentVotersOptimalDecloning}
which uses different clone set error function that \textsc{ComponentOptimalDecloning}.

Table \ref{cclasses} summarizes complexity classes of studied decloning problems.

\begin{table}
\centering
\begin{tabular}{| c | c |}\hline
	\textbf{problem} & \textbf{complexity class} \\ \hline
	\textsc{Decloning} & \p \\ \hline
	\textsc{OptimalDecloning} & \np-complete \\ \hline
	\textsc{ComponentOptimalDecloning} & \np-complete \\ \hline
	\textsc{ConsistentVotersOptimalDecloning} & \np \\ \hline
\end{tabular}
\caption{Complexity classes of considered problems.} \label{cclasses}
\end{table}


\section{Further Works}

Natural continuation of this work is more accurate classification of \textsc{ConsistentVotersOptimalDecloning} problem.
We may expect that, similarly to \textsc{ComponentOptimalDecloning}, this problem is \np-complete.

Moreover, \np-completeness of optimal decloning problems does not mean
that decloning in face of voters' errors is computationally infeasible.
There may be either an efficient approximation algorithm
or a parametrization resulting in fixed-parameter tractable algorithm.

It is also possible that there is a clone set error function for which optimal decloning problem is in \p.
