In order to model views of each voter,
we assume that each voter has a full preference order over the set of candidates.
This model is general enough to describe special cases
when we care only for some first or last candidates for every voter.
It assumes the ability of each voter to arrange all candidates according to his preferences.
On the other hand, it fails to express the situation when a voter perceives that some differences
between his consecutive candidates are bigger than others, e.g., statements such as:
\textit{candidates A and B are fine, with A slightly better than B, but C and D are absolutely not acceptable}.
Range voting \cite{rangevoting} or fallback voting \cite{fallbackvoting} systems
allow for more elastic expression of opinions.

\begin{defn}[preference order]
Given a finite candidate set $C$, a preference order over $C$
is~a~total order over $C$.
\end{defn}

\begin{defn}[preference profile]
Preference profile $P = (\succ_1, ... , \succ_n)$ for given candidate set $C$,
is a sequence of $n$ preference orders over $C$,
where each $\succ_i, 1 \leq i \leq n$, represents the preference order of the $i$-th voter.
\end{defn}

\begin{exmp}
For a candidate set $C = \{a, b, c, d\}$,
let us consider a preference profile $P = (\succ_1, \succ_2, \succ_3)$ with
\begin{align*}
a \succ_1 b \succ_1 c \succ_1 d, \\
b \succ_2 d \succ_2 c \succ_2 a,	\\
a \succ_3 b \succ_3 d \succ_3 c.
\end{align*}
$P$ is a preference profile over $C$.
\end{exmp}

This is where our interests in elections for this work end.
However, for completeness, we will discuss what is election system and give some examples.

\begin{defn}[election]
Election is a pair $(C, P)$, where $C$ is a candidate set, and $P$ is a preference profile for $C$.
\end{defn}

\begin{defn}[election system]
Election system is a function $R : (C,P) \rightarrow 2^{C} \setminus \{\emptyset\}$
that maps election to non-empty set of winners.
\end{defn}

\begin{exmp}[Plurality]
Plurality is an election system that for given election chooses winners as the candidates
with the highest number of voters who ranked them on the first position.
\end{exmp}

\begin{exmp}[Borda]
Borda is an elections system in which each voter gives its $i$-th candidate score $i$.
Winners are candidates with lowest scores.
\end{exmp}

\begin{exmp}[Copeland]
We say for given candidate that a pairwise victory is a situation
when majority of voters prefer that candidate from the other one.
The opposite situation is a pairwise defeat.
Copeland is an election system in which winners are those candidates that have
highest number of pairwise victories minus number of pairwise defeats.
\end{exmp}
