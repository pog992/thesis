In order to model views of each voter we assume that each voter has his full preference order over the set of candidates.
This model is general enough to describe special cases when we care only for some first or last candidates for every voter.
It assumes the ability of each voter to arrange all candidates according to his preferences.
On the other hand, it fails to express the situation when a voter perceives that some differences
between his consecutive candidates are bigger than others, e.g. statement like:
\textit{candidates A and B are fine, with A slightly better than B, but C and D are absolutely not acceptable}.

\begin{defn}[preference order]
Given a finite candidate set $C$ a preference order over $C$
is~a~total order over $C$.
\end{defn}

\begin{defn}[preference profile]
Preference profile $P = (\succ_1, ... , \succ_n)$ for given candidate set $C$
is a sequence of $n$ preference orders over $C$,
where each $\succ_i, 1 \leq i \leq n$ represents preference order of $i$-th voter.
\end{defn}


\begin{exmp}
For a candidate set $C = \{1, 2, 3, 4\}$ we have that $P = (\succ_1, \succ_2, \succ_3)$ with
\begin{align*}
1 \succ_1 2 \succ_1 3 \succ_1 4 \\
2 \succ_2 4 \succ_2 3 \succ_2 1	\\
1 \succ_3 2 \succ_3 4 \succ_3 3
\end{align*}
is a preference profile over $C$.
\end{exmp}