In this chapter we present briefly basic concepts and facts to be used later in the thesis.
It is divided into two sections.
First one is a short revision of necessary mathematics.
Second section revises basics and builds intuition regarding computational complexity theory.
It also introduces some $NP$-complete computational problems for further use.

This chapter might be skipped and used as a reference only when necessary.

\section{Symbols}

% natural, positive natural numbers, integers, real number, sum, ...

\section{Mathematics}

\begin{defn}[total order]
A total order over set $X$ is a binary relation $\succ$ on $X$ which~is:
\begin{itemize}
	\item transitive, i.e. $\forall a,b,c \in X a \succ b \land b \succ c \Rightarrow a \succ c$,
	\item antisymmetric, i.e. $\forall a,b \in X a \succ b \land b \succ a \Rightarrow a=b $,
	\item total, i.e. $\forall a,b \in X a \succ b \lor b \succ a$.
\end{itemize}
\end{defn}

\begin{defn}[transitive closure]
Transitive closure $R^*$ of relation $R$ is a minimal transitive relation 
such that $R^*$ is a superset of $R$.
\end{defn}

\begin{defn}[predicate]
A predicate is a function $f$ with a codomain $\{true, false\}$,
where $true$ is a logical truth and $false$ its negation.
\end{defn}

A multiset is a generalization of the notion of set, it can hold multiple same elements.

\begin{defn}[multiset]
A multiset is a pair $M = (X, c)$, where $X$ is some set and $c:X \rightarrow \mathbb{N}_0$
is a multiplicity function, i.e. $c(x)$ for $x \in X$ denotes the number of occurrences of $x$ in $M$.
\end{defn}

We will say that $x \in M$ when $c(x) > 0$.
We also say that $M_1 = (X, c_1)$ is a multiset sum of $M_2 = (X, c_2)$ and $M_3 = (X, c_3)$
if for each $x \in X$ we have $c_1(x) = c_2(x) + c_3(x)$.

When the set $X$ is clear from the context we will informally call $c$ itself a multiset.

\begin{defn}[set partition]
A partition of a set $X$ is a family $F$ of non-empty subsets of $X$
such that for every $a \in X$ there is exactly one $Y \in F$ such that $a \in Y$.
\end{defn}

Let us now revise some notions from graph theory.

\begin{defn}[graph]
A graph $G$ is an ordered pair $(V,E)$ of set $V$ of nodes
and set $E$ of edges, which are 2-element subsets of $V$.
\end{defn}

\begin{defn}[multigraph]
A multigraph $G$ is an ordered pair $(V,E$) of set $V$ of nodes
and multiset $E$ of edges, which are 2-element subsets of $V$.
\end{defn}

\begin{defn}[induced subgraph]
(Multi)graph $(V', E')$ is a sub(multi)graph of (multi)graph $(V,E)$ induced by $V'$
iff $V' \subset V$, $E' \subset E$ and for each $u,v \in V'$ if $\{u,v\}\in E$ then $\{u,v\}\in E'$.
\end{defn}

\begin{defn}[Hamiltonian path]
A Hamiltionian path in (multi)graph $(V,E)$ where $V = \{u_1, ..., u_n\}$
is a sequence of edges $\{u_1, u_2\}, \{u_2, u_3\} ..., \{u_{n-1}, u_n\}$,
i.e. for each $i=1,...,n-1$ we have $\{u_i, u_{i+1}\} \in E$.
\end{defn}


This theorem states that prime number are dense in an informal sense.

\begin{thm}[Bertrand–Chebyshev]
For any integer $n>1$ there always exists at least one prime $p$ such that $n < p < 2n$.
\end{thm}


The following are definitions necessary to formulate the Karamata's inequality.

\begin{defn}[strictly convex function]
We say that a function $f:\mathbb{R}\rightarrow\mathbb{R}$ is strictly convex on an interval $I \subset \mathbb{R}$
if for each $x_1, x_2 \in I$ and $t \in [0,1]$ occurs $f(t x_1 + (1-t) x_2) < tf(x_1) + (1-t)f(x_2)$.
\end{defn}

The next property is a convenient way of proving that a function is strictly convex.

\begin{prp}
A twice differentiable function $f:\mathbb{R}\rightarrow\mathbb{R}$
that for each $x$ on interval $I \subset \mathbb{R}$ has $f''(x) > 0$
is strictly convex on $I$.
\end{prp}

\begin{defn}[majorization]
We say that sequence $(a_1, ..., a_n) \in \mathbb{R}^n$ majorizes sequence $(b_1, ..., b_n) \in \mathbb{R}^n$
if the following conditions are met:
\begin{itemize}
	\item $a_1 \geq a_2 \geq ... \geq a_n$,
	\item $b_1 \geq b_2 \geq ... \geq b_n$,
	\item $\sum_{i=1}^n a_i = \sum_{i=1}^n b_i$,
	\item $\sum_{i=1}^k a_i \geq \sum_{i=1}^k b_i$ $\forall k \in \{ 1, ..., n\}$.
\end{itemize}
\end{defn}

Finally the Karamata's inequality that formalizes a very intuitive phenomenon.

\begin{thm}[Karamata's inequality] \label{thm:Kar}
If function $f:\mathbb{R}\rightarrow\mathbb{R}$ is strictly convex on interval~$I \subset \mathbb{R}$,
sequence $(a_i) \in I^n$ majorizes sequence $(b_i) \in I^n$ and $(a_i) \neq (b_i)$, then
$\sum_{i=1}^n f(a_i) > \sum_{i=1}^n f(b_i)$.
\end{thm}


\section{Complexity Theory}

We will not give a full introduction to computational complexity theory here.
Instead, we will try to revise the most important notions.

We model an algorithm by an abstract \textit{Turing machine}.
Computation is an execution of Turing machine on a given input.
The Church-Turing conjecture is an informal definition what a Turing machine can.

\begin{thm}[Church-Turing conjecture]
Computability by Turing machine is equivalent to intuitive notion of computability.
\end{thm}

\begin{defn}[decision problem]
Decision problem is any function that transforms an input into \textsc{yes} or \textsc{no}.
\end{defn}

A Turing machine solves a problem if it is its implementation.

\begin{defn}[solvability by Turing machine]
We say that a Turing machine solves a decision problem $A$ if for each input $x$
it ends its execution in state equal to $A(x)$.
\end{defn}

We define \p as a class of problems solvable by a polynomial-time Turing machine.

\begin{defn}[polynomial-time Turing machine]
Polynomial-time Turing machine is such a Turing machine $M$ that there exist integers $k$ and $c$
that for every input of size $n$ machine $M$ executed on this input makes at most $c n^k$ steps.
\end{defn}

\begin{defn}[\p]
Decision problem $A$ belongs to $P$ iff there is a polynomial-time Turing machine that solves $A$.
\end{defn}

A reduction is an algorithm that transforms an input of one problem into another one
preserving the answer.

\begin{defn}[polynomial-time reduction]
Polynomial-time reduction from problem $A$ to $B$ is a polynomial-time Turing machine
that transforms every \textsc{yes}-inputs of $A$ into some \textsc{yes}-input of $B$
and every \textsc{no}-input of $A$ into some \textsc{no}-inputs of $B$.
\end{defn}

If there is a reduction from problem $A$ to $B$, we say that $A$ reduces to $B$.

Here we present some simple facts about reductions.

\begin{rmrk}\label{tran-red}
Relation "reduces polynomially to" is transitive.
\end{rmrk}

\begin{rmrk}\label{p-red}
If problem $A$ reduces polynomially to $B$ and $B \in \p$ then $A \in \p$.
\end{rmrk}

\np is a complexity class of problems to which \textsc{yes}-inputs are polynomially verifiable.

\begin{defn}[\np]
Problem $A$ belongs to \np iff there is a Turing machine $M$ such that
for each \textsc{yes}-input $x$ of $A$ answers \textsc{yes} to input $(x,y)$ for some $y$
and for eaach \textsc{no}-input $z$ of $A$ answers \textsc{no} to every input $(z,w$).
\end{defn}

It remains an open problem whether $\text{\textsc{P}}\neq\np$.
This is why we define \np-hard as complexity class of problems as hard as every problem in \np.

\begin{defn}[\np-hard]
Problem $A$ belongs to \np-hard iff for each problem $B \in \np$
there is a polynomial-time reduction from $B$ to $A$.
\end{defn}

We are often interested only in \np-hard problems that remain in \np class.

\begin{defn}[\np-complete]
Problem $A$ belongs to \np-complete if $A$ belongs both to \np and \np-hard.
\end{defn}

We say that a problem \textit{is} \np-hard or \np-complete if it belongs to \np-hard or \np-complete respectively.

The following captures why proving \np-completeness of a problem is valuable.

\begin{rmrk}
If \np-complete problem $A$ belongs to \p then using remarks \ref{tran-red} and \ref{p-red}
we have that $\p = \np$.
\end{rmrk}

What, using contraposition, gives us that if $\p \neq \np$ then \np-complete problems are not in \p.

We have a very convenient method of proving that a problem is \np-hard.

\begin{rmrk}
If problem $A \in \text{\np-hard}$ reduces polynomially to $B$ then also $B \in \text{\np-hard}$.
\end{rmrk}

With that we only need to show a polynomial-time reduction from a well-known \np-hard problem.
Now we will introduce some such problems.

\begin{problem}{Exact3Cover (X3C)}
	Input: Collection $S$ of three-element subsets of set $X$, $\vert{X}\vert = 3n$.

	Question: Is there an $n$-element subset $A \subset S$ such that $\bigcup A = X$?
\end{problem}

\begin{thm}
\textsc{X3C} is $NP$-complete.
\end{thm}


\begin{problem}{Clique}
    Input: Graph $G = (V, E)$, integer $c$.

    Question: Is there a subgraph of $G$ that is a clique of size $c$?
\end{problem}

\begin{thm}
\textsc{Clique} is $NP$-complete.
\end{thm}


We will show a sample $NP$-completeness proof using the method of reduction.

\begin{problem}{GraphPartition}
Input: Graph $G = (V, E)$, integer $k$ and $l$.

Question: Is it possible to partition $V$ into $k$ parts so that the number of edges
whose vertices belong to the same part is at least $l$?
\end{problem}

We will call an edge whose vertices belong to the same part an internal edge
and the set of all internal edges an intra-part edges.

\begin{thm} \label{gp-np}
\textsc{GraphPartition} is $NP$-complete.
\end{thm}

\begin{proof}
\textsc{GraphPartition} is clearly in $NP$,
as given the partition we may verify its properties in a polynomial time.
To prove $NP$-hardness we will show a reduction from the \textsc{Clique} problem.

We are given graph $G$ and integer $c$.
Let us denote $n = \abs{S}$.
Let us leave $G$ unchanged and take $k = n-c+1$ and $l = c(c-1)/2$.
We will show that this is a valid reduction to \textsc{GraphPartition}.

Let us first consider the case when the answer to the \textsc{Clique} problem is yes,
i.e. $G$ contains clique consisting of vertices $C$, $\abs{C} = c$.
We may then partition $V$ into $k$ parts: $C$ and $n-c$ singletons of each vertex in $V \backslash C$.
The number of intra-part edges equals $c(c-1)/2$ for part $C$ and $0$ for remaining parts.
This gives $l$ in total, so the answer to \textsc{GraphPartition} problem is also yes.

Now it suffices to show that when there is no $c$-sized clique in $G$
the answer to the reduced problem is no.
In that case every $c$-sized part of $V$ has less than $l$ internal edges,
so partitioning $V$ into $k$ parts of sizes $(c, 1, ..., 1)$ will lead to less than $l$ intra-part edges.
Now let us consider partitioning $P$ into parts that form a different sequence $S$ of sizes.
Let us denote the number of intra-part edges in $P$ by $e$.
Every part of size $m$ has at most $f(m) = m(m-1)/2$ internal edges,
this number is reached only when a part forms a clique.
Function $f$ is strictly convex, because $f'' = 1 > 0$ and $(c,1,...,1)$ majorizes $s$,
so we can use inequality \ref{thm:Kar} for $f$ and both sequences.
This gives us the following upper limit on $e$:
$$e \leq \sum_{s\in{S}} f(s) < f(c) + f(1) + ... + f(1) = l$$
We thus get that each partition has less than $l$ internal edges,
so the answer to the \textsc{GraphPartition} problem remains no.
\end{proof}

% circuit
