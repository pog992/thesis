In this chapter we present briefly basic concepts and facts to be used later in the thesis.
It is divided into three sections.
First one is a short revision of necessary mathematics.
Second section build intuition regarding computational complexity theory.
Finally in the last section we formally introduce notions connected to cloning in elections.

\section{Mathematics}

\begin{defn}
A total order over set $X$ is a binary relation $\succ$ on $X$ which~is:
\begin{itemize}
	\item transitive, i.e. $\forall a,b,c \in X a \succ b \land b \succ c \Rightarrow a \succ c$,
	\item antisymmetric, i.e. $\forall a,b \in X a \succ b \land b \succ a \Rightarrow a=b $,
	\item total, i.e. $\forall a,b \in X a \succ b \lor b \succ a$.
\end{itemize}
\end{defn}


\begin{defn}
A partition of a set $X$ is a family $F$ of non-empty subsets of $X$
such that for every $a \in X$ there is exactly one $Y \in F$ such that $a \in Y$.
\end{defn}


\begin{defn}
A graph $G$ is an ordered pair $(V,E)$ of set $V$ of nodes
and set $E$ of edges, which are 2-element subsets of $V$.
\end{defn}

\begin{defn}
A multigraph $G$ is an ordered pair $(V,E$) of set $V$ of nodes
and multiset $E$ of edges, which are 2-element subsets of $V$.
\end{defn}

%\begin{defn}
%A path in a graph or multigraph $G=(V,E)$ is a 
%\end{defn}

%\begin{defn}
%A Hamiltonian path is ...
%\end{defn}


\begin{thm}[Bertrand–Chebyshev]
For any integer $n>1$ there always exists at least one prime $p$ such that $n < p < 2n$.
\end{thm}



\begin{defn}
We say that a function $f:\mathbb{R}\rightarrow\mathbb{R}$ is convex on an interval $I \subset \mathbb{R}$
if for each $x_1, x_2 \in I$ and $t \in [0,1]$ occurs $f(t x_1 + (1-t) x_2) \leq tf(x_1) + (1-t)f(x_2)$.
\end{defn}

\begin{prp}
A twice differentiable function $f:\mathbb{R}\rightarrow\mathbb{R}$ is convex on an interval $I \subset \mathbb{R}$
if and only if for each $x \in I$ occurs $f''(x) \geq 0$.
\end{prp}

\begin{defn}
We say that sequence $(a_1, ..., a_n) \in \mathbb{R}^n$ majorizes sequence $(b_1, ..., b_n) \in \mathbb{R}^n$
if the following conditions are met:
\begin{itemize}
	\item $a_1 \geq a_2 \geq ... \geq a_n$,
	\item $b_1 \geq b_2 \geq ... \geq b_n$,
	\item $\sum_{i=1}^n a_i = \sum_{i=1}^n b_i$,
	\item $\sum_{i=1}^k a_i \geq \sum_{i=1}^k b_i$ $\forall k \in \{ 1, ..., n\}$.
\end{itemize}
\end{defn}

\begin{thm}[Karamata's inequality]
If $f:\mathbb{R}\rightarrow\mathbb{R}$ is convex on interval~$I$
and $(a_i) \in I^n$ majorizes $(b_i) \in I^n$ then
$\sum_{i=1}^n f(a_i) \geq \sum_{i=1}^n f(b_i)$.
\end{thm}


\section{Complexity Theory}

% p, np, np-completness

\section{Elections} \input{"elections.tex"}