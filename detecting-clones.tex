\documentclass[a4paper,12pt]{report}
\usepackage[utf8]{inputenc}
%\usepackage{polski}
%\usepackage[polish]{babel}
\usepackage[T1]{fontenc}
%\usepackage[latin2]{inputenc} 
\usepackage{fullpage}
\linespread{1.3}

% generate index
\usepackage[hidelinks]{hyperref}

\usepackage[nottoc]{tocbibind}

%\usepackage[unicode]{hyperref}

%\usepackage{url}
%\usepackage{listings}

%\usepackage{indentfirst}
%\setcounter{secnumdepth}{0}

\usepackage{amsmath}
%\newcommand{\card}[1]{\ensuremath{\left\|#1\right\|}}
\usepackage{amssymb}
\usepackage{amsthm}

\usepackage{xspace}

\makeatletter
% \def\@endtheorem{\endtrivlist\@endpefalse }% OLD
\def\@endtheorem{\endtrivlist}% NEW
\makeatother

\theoremstyle{plain}
\newtheorem{thm}{Theorem}[chapter] % reset theorem numbering for each chapter
\newtheorem{prp}[thm]{Property}
\newtheorem{lmm}[thm]{Lemma}
\newtheorem{cnj}[thm]{Conjecture}

\theoremstyle{definition}
\newtheorem{defn}[thm]{Definition} % definition numbers are dependent on theorem numbers
\newtheorem{exmp}[thm]{Example} % same for example numbers

\theoremstyle{remark}
\newtheorem{rmrk}[thm]{Remark}

\newcommand{\abs}[1]{\left\vert #1 \right\vert}
%\newcommand{\np}{\textsc{NP}}
\newcommand{\np}{\text{\textsc{NP}}\xspace}
\newcommand{\p}{\text{\textsc{P}}\xspace}

\author{Szymon Gut\\\newline\\supervisor:\\dr hab. inż. Piotr Faliszewski}

\title{\textsc{The Computational Complexity of~Detecting Clones in Elections}}
\date{}


\usepackage{environ}

\NewEnviron{problem}[1]{%
\begin{center}\fbox{\parbox{\linewidth}{%
    {\centering\scshape #1\par}%
    \parskip=1ex
    \everypar{\hangindent=1em}%
    \BODY
}}\end{center}}

\begin{document}

\maketitle


\abstract{
In this work we study computational complexity of problems related to elections.
We are interested in problems of detecting whether
some subsets of candidates are either indistinguishable or hardly distinguishable for voters.
We call such candidates clones and imperfect clones, respectively.
We conclude that detecting clones is in \textsc{P}.
For imperfect clones we discuss measures of the distance from some clone.
We define optimal decloning as the problem of finding imperfect clones
with minimal sum of the distances.
We show that optimal decloning is \textsc{NP}-complete for some measures.
}

\tableofcontents

\chapter{Introduction} \input{"introduction.tex"}
\chapter{Preliminaries} \input{"preliminaries.tex"}
\chapter{Clones in Elections} \input{"decloning.tex"}
\chapter{Imperfect Clones} \input{"imperfect.tex"}
\chapter{Summary} \input{"summary.tex"}


%\nocite{*}
\bibliographystyle{plain}
\bibliography{bib}

\end{document}
