\documentclass[a4paper,12pt]{report}
\usepackage[utf8]{inputenc}
%\usepackage{polski}
%\usepackage[polish]{babel}
\usepackage[T1]{fontenc}
%\usepackage[latin2]{inputenc} 
\usepackage{fullpage}
\usepackage{setspace}
\linespread{1.3}

% generate index
\usepackage[hidelinks]{hyperref}

\usepackage[nottoc]{tocbibind}

%\usepackage[unicode]{hyperref}

%\usepackage{url}
%\usepackage{listings}

%\usepackage{indentfirst}
%\setcounter{secnumdepth}{0}

\usepackage{amsmath}
%\newcommand{\card}[1]{\ensuremath{\left\|#1\right\|}}
\usepackage{amssymb}
\usepackage{amsthm}

\usepackage{xspace}

\usepackage{graphicx}

\makeatletter
% \def\@endtheorem{\endtrivlist\@endpefalse }% OLD
\def\@endtheorem{\endtrivlist}% NEW
\makeatother

\theoremstyle{plain}
\newtheorem{thm}{Theorem}[chapter] % reset theorem numbering for each chapter
\newtheorem{prp}[thm]{Property}
\newtheorem{lmm}[thm]{Lemma}
\newtheorem{cnj}[thm]{Conjecture}

\theoremstyle{definition}
\newtheorem{defn}[thm]{Definition} % definition numbers are dependent on theorem numbers
\newtheorem{exmp}[thm]{Example} % same for example numbers

\theoremstyle{remark}
\newtheorem{rmrk}[thm]{Remark}

\newcommand{\abs}[1]{\left\vert #1 \right\vert}
%\newcommand{\np}{\textsc{NP}}
\newcommand{\np}{\text{\textsc{NP}}\xspace}
\newcommand{\p}{\text{\textsc{P}}\xspace}

\author{Szymon Gut\\\newline\\supervisor:\\dr hab. inż. Piotr Faliszewski}

\title{\textsc{The Computational Complexity of~Detecting Clones in Elections}}
\date{}


\usepackage{environ}

\NewEnviron{problem}[1]{%
\begin{center}\fbox{\parbox{\linewidth}{%
    {\centering\scshape #1\par}%
    \parskip=1ex
    \everypar{\hangindent=1em}%
    \BODY
}}\end{center}}

\begin{document}

\begin{titlepage}
\begin{center}
{\Large AGH University of Science and Technology}\\[0.2cm]
{\Large Faculty of Computer Science,\\Electronics and Telecommunications}\\[0.4cm]
{\Large Department of Computer Science}\\[1.5cm]

{\Large Master's Thesis}\\[1.0cm]

\begin{spacing}{2}
\textsc{\huge The Computational Complexity\\of~Detecting Clones in Elections}\\[3cm]
\end{spacing}

\end{center}

\vfill
\begin{flushleft}
\begin{tabular}{p{45mm}l}
\large Author: & {\large \itshape Szymon Gut}\\[-1mm]
\large Degree programme: & {\large \itshape Computer Science}\\[-1mm]
\large Supervisor: & {\large \itshape dr hab. inż. Piotr Faliszewski}\\[2.0cm]
\end{tabular}
\end{flushleft}

\begin{center}
Kraków, 2015
\end{center}

\clearpage

\sloppy
Oświadczam, świadomy odpowiedzialności karnej za poswiadczenie nieprawdy,
że~niniejszą pracę dyplomową wykonałem osobiście i samodzielnie i nie korzystałem
ze~źródeł innych niż wymienione w pracy.


\end{titlepage}

%\maketitle


\abstract{
In this work we study the computational complexity of problems related to elections.
We are interested in problems of detecting whether some subsets of candidates
are either indistinguishable or hardly distinguishable from the point of view of the voters.
We call such candidates clones and imperfect clones, respectively.
We treat detecting clones as the problem of partitioning the candidate set
into subsets whose elements are ranked consecutively by every voter.
We conclude that detecting clones is a polynomial-time solvable task.
For imperfect clones we discuss measures of the distance from being some clone.
We define optimal decloning as the problem of finding imperfect clones
with minimal sum of the distances.
We show that optimal decloning is \textsc{NP}-complete for some natural measures.
}

\tableofcontents

\chapter{Introduction} \input{"introduction.tex"}
\chapter{Preliminaries} \input{"preliminaries.tex"}
\chapter{Clones in Elections} \input{"decloning.tex"}
\chapter{Imperfect Clones} \input{"imperfect.tex"}
\chapter{Summary} \input{"summary.tex"}


%\nocite{*}
\bibliographystyle{plain}
\bibliography{bib}

\end{document}
