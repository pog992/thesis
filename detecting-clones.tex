\documentclass[a4paper,12pt]{report}
\usepackage[utf8]{inputenc}
%\usepackage{polski}
%\usepackage[polish]{babel}
\usepackage[T1]{fontenc}
%\usepackage[latin2]{inputenc} 
\usepackage{fullpage}
\linespread{1.3}

% generate index
\usepackage[hidelinks]{hyperref}

\usepackage[nottoc]{tocbibind}

%\usepackage[unicode]{hyperref}

%\usepackage{url}
%\usepackage{listings}

%\usepackage{indentfirst}
%\setcounter{secnumdepth}{0}

\usepackage{amsmath}
%\newcommand{\card}[1]{\ensuremath{\left\|#1\right\|}}
\usepackage{amssymb}
\usepackage{amsthm}

\makeatletter
% \def\@endtheorem{\endtrivlist\@endpefalse }% OLD
\def\@endtheorem{\endtrivlist}% NEW
\makeatother

\theoremstyle{plain}
\newtheorem{thm}{Theorem}[chapter] % reset theorem numbering for each chapter
\newtheorem{prp}[thm]{Property}
\newtheorem{lmm}[thm]{Lemma}

\theoremstyle{definition}
\newtheorem{defn}[thm]{Definition} % definition numbers are dependent on theorem numbers
\newtheorem{exmp}[thm]{Example} % same for example numbers

\theoremstyle{remark}
\newtheorem{rmrk}[thm]{Remark}

\newcommand{\abs}[1]{\left\vert #1 \right\vert}
%\newcommand{\np}{\textsc{NP}}
\newcommand{\np}{\text{\textsc{NP}}}
\newcommand{\p}{\text{\textsc{P}}}

\author{Szymon Gut\\\newline\\supervisor:\\dr hab. inż. Piotr Faliszewski}

\title{\textsc{The Computational Complexity of~Detecting Clones in Elections}}
\date{}


\usepackage{environ}

\NewEnviron{problem}[1]{%
\begin{center}\fbox{\parbox{\linewidth}{%
    {\centering\scshape #1\par}%
    \parskip=1ex
    \everypar{\hangindent=1em}%
    \BODY
}}\end{center}}

\begin{document}

\maketitle


\abstract{
In elections, a subset of candidates ranked consecutively (though possibly in~different order)
by all voters is called a clone set, and its members are called clones.
A clone structure is a family of all clone sets of a given election.
In~this work we study the problem of decloning, i.e. finding the clone structure of given cardinality.
We give a polynomial-time algorithm for decloning problem.
We also consider decloning into imperfect clone structure: we define error function measuring
cost of turning given clone sets family into clone structure.
Finally, we prove that optimal decloning for~sample error function is $NP$-complete.
}

\tableofcontents

\chapter{Introduction} \input{"introduction.tex"}
\chapter{Preliminaries} \input{"preliminaries.tex"}
\chapter{Clones in Elections} \input{"decloning.tex"}
\chapter{Imperfect Clones} \input{"imperfect.tex"}
\chapter{Summary} \input{"summary.tex"}


\nocite{*}
\bibliographystyle{plain}
\bibliography{bib}

\end{document}