\documentclass{beamer}
%\usetheme{Warsaw}
\useoutertheme{split}
\useinnertheme{rounded}
\usecolortheme{crane}
\beamertemplatenavigationsymbolsempty

\usepackage[utf8]{inputenc}
\usepackage{polski}
\usepackage[polish]{babel}
\usepackage{graphicx}
\usepackage{listings}
\usepackage{amsthm}
\usepackage{wrapfig}

\usepackage{environ}

\NewEnviron{aproblem}[1]{%
\begin{center}\fbox{\parbox{\linewidth}{%
    {\centering\scshape #1\par}%
    \parskip=1ex
    \everypar{\hangindent=1em}%
    \BODY
}}\end{center}}


\begin{document}	
\title[Złożoność obliczeniowa wykrywania klonów w wyborach]{Złożoność obliczeniowa\\ wykrywania klonów w wyborach}
\author[Szymon Gut]{Szymon Gut\\ \hfill \\opiekun pracy:\\dr hab. inż. Piotr Faliszewski}
%\institute{AGH}
\date{}
\frame{\titlepage}

%\begin{frame} \frametitle{Spis treści}
%		\tableofcontents
%\end{frame}

\begin{frame} \frametitle{Podejście}
  obrona przed atakami na systemy wyborcze przez wykrywanie ich

  jeśli wykrywanie 
\end{frame}

\begin{frame} \frametitle{Atak przez klonowanie}

\end{frame}

\begin{frame} \frametitle{Wybory}
  zbiór kandydatów $C$
\end{frame}

\begin{frame} \frametitle{Klony}

\end{frame}

\begin{frame} \frametitle{Deklonowanie}
  \begin{aproblem}{Deklonowanie}
    Wejście: profil preferencji $P$ nad zbiorem kandydatów $C$, liczba naturalna $k$

    Pytanie: czy istnieje deklonowanie dla $P$ o mocy $k$?
  \end{aproblem}

  $\textsc{Deklonowanie} \in \text{P}$

\end{frame} 

\begin{frame} \frametitle{Niedoskonałe klony}
  przyjmujemy funkcję błędu $f: 2^C \setminus \{\emptyset\} \rightarrow \mathbb{R}_0^+$,
  która mierzy błąd danego niepustego podzbioru kandydatów 
\end{frame}

\begin{frame} \frametitle{Wnioski}
  
\end{frame}

\frame{\titlepage}

\end{document}
