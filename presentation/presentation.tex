\documentclass{beamer}
%\usetheme{Warsaw}
\useoutertheme{split}
\useinnertheme{rounded}
\usecolortheme{crane}
\beamertemplatenavigationsymbolsempty

\usepackage[utf8]{inputenc}
\usepackage{polski}
\usepackage[polish]{babel}
\usepackage{graphicx}
\usepackage{listings}
\usepackage{amsthm}
\usepackage{wrapfig}
\usepackage{subcaption}
\usepackage{marvosym}
\usepackage{color}

\usepackage{environ}

\NewEnviron{aproblem}[1]{%
\begin{center}\fbox{\parbox{\linewidth}{%
    {\centering\scshape #1\par}%
    \parskip=1ex
    \everypar{\hangindent=1em}%
    \BODY
}}\end{center}}


\begin{document}	
\title[Złożoność obliczeniowa wykrywania klonów w wyborach]{Złożoność obliczeniowa\\ wykrywania klonów w wyborach}
\author[Szymon Gut]{Szymon Gut\\ \hfill \\opiekun pracy:\\dr hab. inż. Piotr Faliszewski}
%\institute{AGH}
\date{}
\frame{\titlepage}

%\begin{frame} \frametitle{Spis treści}
%		\tableofcontents
%\end{frame}


\begin{frame} \frametitle{Atak przez klonowanie}
	wprowadzenie do wyborów nowego kandydata (kandydatów), bardzo podobnego do pewnego istniejącego
	
	\vfill

	\begin{center}
	\begin{figure}
        \centering
        \begin{subfigure}[b]{0.3\textwidth}
			\begin{tabular}{ | l |  r | } \hline
  				$\mathbf{A}$ & \textbf{60\%} \\ \hline
  				$B$ & 40\% \\ \hline
			\end{tabular}
        \end{subfigure}
        ~ %add desired spacing between images, e. g. ~, \quad, \qquad, \hfill etc.
          %(or a blank line to force the subfigure onto a new line)
        \begin{subfigure}[b]{0.1\textwidth}
        	\MVRightarrow
        \end{subfigure}
        \begin{subfigure}[b]{0.3\textwidth}
			\begin{tabular}{ | l |  r | } \hline
  				$A$ & 35\% \\ \hline
  				$A'$ & 25\% \\ \hline
  				$\mathbf{B}$ & \textbf{40\%} \\ \hline
			\end{tabular}
        \end{subfigure}
	\end{figure}
	\end{center}
\end{frame}

\begin{frame} \frametitle{Podejście}
	\begin{itemize}
		\item obrona przed atakiem przez wykrycie go
										
		\vfill
		\item jeśli wykrywanie da się przeprowadzić efektywnie,\\
				 atak uznajemy za mało groźny
		\item w przeciwnym razie, atak może być niebezpieczny
	\end{itemize}
\end{frame}

\begin{frame} \frametitle{Wybory}
  zbiór kandydatów $C$, \\ 
  pełne preferencje wyborcze każdego wyborcy (porządek liniowy nad zbiorem kandydatów), mamy tzw. profil preferencji $P$

	\vfill

	Przykład: $C = \{a, b, c, d\}$, $P = (\succ_1, \succ_2, \succ_3, \succ_4)$
   	\begin{align*}
		a \succ_1 b \succ_1 c \succ_1 d \succ_1 e \succ_1 f \\
		b \succ_2 c \succ_2 a \succ_2 d \succ_2 f \succ_2 e \\
		e \succ_3 f \succ_3 c \succ_3 b \succ_3 a \succ_3 d \\
		d \succ_4 f \succ_4 e \succ_4 a \succ_4 c \succ_4 b
	\end{align*}
\end{frame}

\begin{frame} \frametitle{Klony}
	\begin{align*}
		\textcolor{blue}{a \succ_1 b \succ_1 c} \succ_1 \textcolor{red}{d} \succ_1 \textcolor{green}{e \succ_1 f} \\
		\textcolor{blue}{b \succ_2 c \succ_2 a} \succ_2 \textcolor{red}{d} \succ_2 \textcolor{green}{f \succ_2 e} \\
		\textcolor{green}{e \succ_3 f} \succ_3 \textcolor{blue}{c \succ_3 b \succ_3 a} \succ_3 \textcolor{red}{d} \\
		\textcolor{red}{d} \succ_4 \textcolor{green}{f \succ_4 e} \succ_4 \textcolor{blue}{a \succ_4 c \succ_4 b}
	\end{align*}
\end{frame}

\begin{frame} \frametitle{Deklonowanie}
  \begin{aproblem}{Deklonowanie}
    Wejście: profil preferencji $P$ nad zbiorem kandydatów $C$, liczba naturalna $k$

    Pytanie: czy istnieje deklonowanie dla $P$ o mocy $k$?
  \end{aproblem}

  $\textsc{Deklonowanie} \in \text{P}$

\end{frame} 

\begin{frame} \frametitle{Niedoskonałe klony}
  przyjmujemy funkcję błędu $f: 2^C \setminus \{\emptyset\} \rightarrow \mathbb{R}_0^+$,
  która mierzy błąd danego niepustego podzbioru kandydatów 
\end{frame}

\begin{frame} \frametitle{Wyniki}


\begin{tabular}{| c | c |}\hline
	\textbf{problem} & \textbf{klasa zł. obl.} \\ \hline
	\textsc{Deklonowanie} & P \\ \hline
	\textsc{OptymalneDeklonowanie} & NP-complete \\ \hline
	\textsc{KomponentoweOptymalneDeklonowanie} & NP-complete \\ \hline
	\textsc{ConsistentVotersOptimalDecloning} & NP \\ \hline
\end{tabular}

\end{frame}

\frame{\titlepage}

\end{document}
