
% preference list motivation


%Opinion aggregation is crucial in contemporary societes.
%Ubiquitois democracy
%Group decision making is crucial 

\section{Attacking Elections with Clones}


Let us imagine single-round presidential elections.
According to sounding there are two considerable candidates: $A$ and $B$.
Candidate $B$ is given lesser chances of winning.
However, he exploits the fact that $A$ has a twin brother $C$.
He manages to persuade him to take part in elections.
When it comes to the elections confused supporters of $A$ do not know on whom they should vote.
Their votes are being split between $A$ and $C$.
Candidate $B$ wins the elections and becomes the president.

This is a particular case of attack called \textit{strategic candidacy problem} or \textit{control by adding candidates}
depending on the executor.
We are interested in variant where new candidate is very similar to an existing one.
So similar, that he or she is indistinguishable or hardly distinguishable for voters.

% move to the beginning
Our goal in this paper is to conceive methods of defending against attacking elections with clones.
We are interested in stating whether such an attack took place.
Given voters preferences, regardless of election system, we want to detect latent clone structures.

%\section{Related Works}


\section{The Structure of the Thesis}

% list. highlight own work.

This thesis has the following structure.
Chapter 2 presents briefly basic mathematical and computational scientific concepts and facts to be used later.
Chapter 3 formalizes definitions of elections and clones.
Chapter 4 introduces decloning problem and gives polynomial-time algorithm for it.
Chapter 5 contains study of imperfect clone structures:
it discusses error function for clone set families and optimal non-perfect decloning problem.
For a sample error function we prove that optimal decloning is \textit{NP}-complete.
Finally, section 6 summarizes thesis results, gives real-life consequences and indicates opportunities for further works.