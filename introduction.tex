Elections is a situation when a set of voters collaboratively chooses one or many winners from multiple candidates.
It is an established way of aggregating opinions into a single global decision.
Group decision making is crucial in contemporary societies.
It may be used also to reach consensus in distributed computer systems.

\section{Attacking Elections with Clones}

Let us imagine a single-round presidential election.
According to preelection polls there are two serious candidates: $A$ and $B$.
Candidate $B$ is given lesser chances of winning.
However, he exploits the fact that $A$ has a twin brother $C$.
He manages to persuade the twin brother to take part in the election.
When it comes to the election, confused supporters of $A$ do not know who they should vote for.
Their votes end up split between $A$ and $C$.
Candidate $B$ wins the elections and becomes the president.

This is a particular case of an attack known as \textit{strategic candidacy} or \textit{control by adding candidates}, %TODO: citation
depending on the person performing the attack.
We are interested in the variant where the new candidates are very similar to some of the existing ones.
So similar, that they are indistinguishable or hardly distinguishable for voters.
We call such candidates clones.

Our goal in this work is to conceive methods of defending against such clone-based attacks.
We are interested in figuring out whether such an attack took place, or, at least, likely took place.
Given voters preferences, regardless of the election system used, we want to detect latent clone structures.

\section{Related Works}


\section{Structure of the Thesis}

This thesis has the following structure.
In Chapter 2 we briefly present concepts from computer science
and mathematics that form the basis of our arguments.
In Chapter 3 we include formal definitions of elections, introduce the decloning problem
and show a polynomial-time algorithm for it.
Chapter 4 contains study of imperfect clone structures.
It discusses error functions for clone set families and the optimal non-perfect decloning problems.
For a sample error function we prove that optimal decloning is \np-complete.
Finally, Chapter 5 summarizes our results, discusses real-life consequences,
and indicates opportunities for further research.
