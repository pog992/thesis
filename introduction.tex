Holding an election is a situation where a set of voters collaboratively chooses
one or many winners among multiple candidates.
It is an established way of aggregating opinions of multiple agents into a single global decision.
Group decision making is crucial in contemporary societies.
It is used in both formal election systems, such as parliamentary elections,
and also in informal casual situation, such as deciding on the date of meeting that suits most interested parties.
It may be used also to reach consensus in distributed computer systems, e.g. as in the Paxos algorithm \cite{paxos}.

\section{Attacking Elections with Clones}

Let us imagine a single-round presidential election.
According to preelection polls there are two serious candidates: $A$ and $B$.
Candidate $B$ is given lesser chances of winning.
However, he exploits the fact that $A$ has a twin brother $C$.
He manages to persuade the twin brother to take part in the election.
When it comes to the election, confused supporters of $A$ do not know who they should vote for.
Their votes end up split between $A$ and $C$.
Candidate $B$ wins the election and becomes the president.

This is a particular case of an attack known as \textit{strategic candidacy},
studied by Dutta et al. \cite{strategiccandidacy},
or \textit{control by adding candidates}, studied by Bartholdi et al. \cite{controlbyadding},
depending on the person performing the attack.
We are interested in the variant where the new candidates are very similar to some of the existing ones.
So similar, that they are indistinguishable or hardly distinguishable for voters.
We call such candidates clones.
Clones can be deliberately introduced to election to alter its outcome.

Our goal in this work is to assess how dangerous are such clone-based attacks.
We aim to figure out whether such an attack took place, or, at least, likely took place.
Given voters preferences, regardless of the election system used, we want to detect latent clone structures.
In this work we study computational complexity of detection problems.
We treat easily-computable detection as an argument that it is easy to defend against clone-based attacks.
On the other hand, hard-to-compute detection means that such attacks are a serious threat.

\section{Related Work}

The theory of elections is a well-studied research area.
Arrow et al. \cite{handbook} presents general overview of the social choice theory.
It describes, among others, Arrow's Impossibility Theorem.
It states that no election system can simultaneously satisfy a set of specific fairness criteria.

Computational approach to elections has been described in the work of
Brandt et al. \cite{compsocialchoice}.
They view the social choice theory from the algorithmic perspective.
It studies the problem of strategic manipulation
and circumventing impossibilities pointed by the Gibbard–Satterthwaite Theorem.
The theorem states that a voting rule that is deterministic and not dictatorial is susceptible to tactical voting
-- situation when a voter can vote not accordingly to his sincere preference in order to prevent an undesirable outcome.

Faliszewski et al. \cite{usingcomplexity} discuss defending against attacks on elections using computational complexity:
making the attack computationally prohibitive.
For some election systems it was proven that the algorithms of seeking the attack on the elections are \np-hard.
The work studied also dichotomy theorems for elections systems to be computationally resistant to attacks.

More background on computational-complexity approach to elections can be found
in the work of Bartholdi and Orlin \cite{usingcomp1}.
They studied computational resilience to attacks of the Single Transferable Vote election system.
They showed that the problem of finding a preference that makes given candidate win the election is \np-complete.
Another work of Bartholdi et al. \cite{usingcomp2} shows an election system,
where computing winners can be done efficiently, but strategic manipulation is computationally prohibitive.

Bartholdi et al. \cite{controlbyadding} studied the computational complexity of altering the election
outcome by introducing new candidates.
This is the situation we deal with when elections are attacked by introducing clones.

Structures of groups of simillar candidates have been studied for instance by
Laffond et al. \cite{clones1}, and Laslier \cite{clones2, clones3}.
Tideman \cite{independenceofclones} discusses an attack on elections where new candidates,
very similar to existing ones, clones, are introduced to change the outcome of an election.
He formulates a condition that makes voting rules resilient to such attacks.

Algorithms for cloning for various voting rules are studied by Elking et al. \cite{cloninginelections}.
They characterize the preference profiles for which exists successful attack by cloning.
They also study the problem of finding minimum-cost attack for a model with weighted cost of producing each clone.

Elkind et al. \cite{clonestructures} give properties of clone structures, regardless of the voting rule.


\section{Structure of the Thesis}

This thesis has the following structure.
In Chapter 2 we briefly present concepts from computer science
and mathematics that form the basis of our arguments.
In Chapter 3 we include formal definitions of elections, introduce the \textsc{Decloning} problem
and show a polynomial-time algorithm for it.
Chapter 4 contains study of imperfect clone structures.
It discusses error functions for clone set families and the optimal non-perfect decloning problems.
For a sample natural error function we prove that optimal decloning is \np-complete.
Finally, Chapter 5 summarizes our results, discusses real-life consequences,
and indicates opportunities for further research.
